\documentclass{article}


% if you need to pass options to natbib, use, e.g.:
%     \PassOptionsToPackage{numbers, compress}{natbib}
% before loading neurips_2023


% ready for submission
\usepackage[preprint]{neurips_2023}


% to compile a preprint version, e.g., for submission to arXiv, add add the
% [preprint] option:
%     \usepackage[preprint]{neurips_2023}


% to compile a camera-ready version, add the [final] option, e.g.:
%     \usepackage[final]{neurips_2023}


% to avoid loading the natbib package, add option nonatbib:
%    \usepackage[nonatbib]{neurips_2023}


\usepackage[utf8]{inputenc} % allow utf-8 input
\usepackage[T1]{fontenc}    % use 8-bit T1 fonts
\usepackage{hyperref}       % hyperlinks
\usepackage{url}            % simple URL typesetting
\usepackage{booktabs}       % professional-quality tables
\usepackage{amsfonts}       % blackboard math symbols
\usepackage{nicefrac}       % compact symbols for 1/2, etc.
\usepackage{microtype}      % microtypography
\usepackage{xcolor}         % colors

\title{Theory and Methods for the Ferromagnetic Ising Model}


% The \author macro works with any number of authors. There are two commands
% used to separate the names and addresses of multiple authors: \And and \AND.
%
% Using \And between authors leaves it to LaTeX to determine where to break the
% lines. Using \AND forces a line break at that point. So, if LaTeX puts 3 of 4
% authors names on the first line, and the last on the second line, try using
% \AND instead of \And before the third author name.


\author{
    Jay Shen \\
    Department of Physics \\
    University of Chicago\\
    Chicago, IL 60637 \\
    \texttt{jshe@uchicago.edu} \\
    \And
    Mark Lee \\
    Department of Statistics \\
    University of Chicago\\
    Chicago, IL 60637 \\
    \texttt{markyl@uchicago.edu} \\
}


\begin{document}

\maketitle

\begin{abstract}

The Ising model is a historically important problem in statistical mechanics. 
Although it was originally proposed as a crude approximation of ferromagnetic 
phenomena, it furnished statistical mechanics, and later probabilistic science 
as whole, with a new paradigm of graph-based modeling that would prove 
influential.
Especially in the distilled, theoretical study of probabilistic graphical 
models (PGMs), which today is its own field, many methods developed for Ising models 
and spin-glasses have been repurposed and reinterpreted, as have the rich 
physical vocabulary of energies, entropy, and partition functions. 

In this paper, we pay homage to the Ising model by applying modern methods to 
the original problem of ferromagnetism. 
We discuss the theory of the Ising model both from a PGMs and statistical 
mechanics point of view, examining the correspondence. 
We then evaluate a number of approaches to inference—Markov Chain Monte Carlo, 
belief propagation, and variational inference. 
Finally, we show the successes and limitations of the Ising model in the context 
of physical phenomena. 

\end{abstract}


\section{Theory of the Ising Model}

A particle with magnetic moment $\vec{\mu}$ in an external magnetic field 
$\vec{B}$ has potential energy:
\[U = - \vec{\mu} \cdot \vec{B}\]
A particle also creates an instrinsic magnetic field defined by magnetic moment 
$\vec{\mu}$. 
The pairwise interactions between particles, described by exchange constants 
$J_{ij}$, thus also have potential energy:
\[U = - J_{ij} \vec{\mu_1} \cdot \vec{\mu_2}\]
We will ignore the short-range Pauli exclusion interactions. 

Now, consider a collection of particles $\vec{\mu_i}$ in the prescence of an 
external magnetic field $\vec{B}$. 
The Hamiltonian, which specifies the total energy of the system, is defined by 
the sum of all pairwise and all magnetic field energies:
\begin{equation}\label{exactE}
    E = -\frac{1}{2}\sum_i \sum_j J_{ij} \vec{\mu_i} \cdot \vec{\mu_j} - \sum_i \vec{\mu_i} \cdot \vec{B}
\end{equation}
In most cases, working with this Hamiltonian is intractable. 
Luckily, we can simplify it by making several strategic assumptions. 

First, we assume all atoms are fermionic spin-$\frac{1}{2}$ and identical. 
Then the magnetic moments simplify to $\vec{\mu_i} = \mu \sigma_i$, where 
$\sigma_i \in \{-1, 1\}$ is the spin of the atom. 

Second, we make a mean-field approximation and assume that the atoms are 
arranged in a regular lattice, and that the strength of pairwise interactions 
are negligible for non-adjacent pairs.

Coalescing constants, the Hamiltonian reduces to:
\begin{equation}\label{isingE}
    E = -J\sum_i \sum_{j \in adj(i)} \sigma_i \sigma_j - \sum_i \sigma_i B_i
\end{equation}
Following the Boltzmann distribution, the probability of some state 
$\vec{\sigma}$ at inverse temperature $\beta$ is:
\begin{equation} \label{boltzmann}
    P(\vec{\sigma}) = \frac{1}{Z}e^{-\beta E(\vec{\sigma})} 
\end{equation}

\section{Inference on Ising Models}

In most cases, we are interested in maximum a posteriori (MAP) inference on 
Ising models, namely, the most likely instantiation of variables. 
In the context of the physical interpretation, this corresponds to finding the 
equilibrium distribution of spins that minimize the energy. 

We will now discuss some approaches to MAP inference on Ising models. 

\subsection{Sample Based}

Sample based Monte Carlo methods aim to discover a MAP estimate by repeated 
sampling from the distribution. 
Markov Chain Monte Carlo (MCMC) improves on this approach by, at each step, 
conditioning the distribution on all previous samples and defining a Markov 
chain of states and transition probabilities. 

In the context of the Ising model, sampling from the conditioned distribution 
corresponds to flipping spins of some spin state $\vec{\sigma}$ to get some 
new state $\vec{\sigma}'$. 
Intuitively, the transition to state $\vec{\sigma}'$ is favorable if 
$P(\vec{\sigma}') > P(\vec{\sigma})$. 
Thus, we define the transition "probability" to be:
\[
    P_{\vec{\sigma} \rightarrow \vec{\sigma}'}
    = \frac{P(\vec{\sigma}')}{P(\vec{\sigma})}
    = e^{-\beta(E(\vec{\sigma}') - E(\vec{\sigma}))} 
\]
In the case where only one spin $\sigma_i$ flips, the energy difference in the 
exponential is:
\[
    E(\vec{\sigma}') - E(\vec{\sigma}) = 2 \sigma_i (J \sum_{j \in adj(i)} \sigma_j + B_i)
\]
So, we get:
\[
    P_{\vec{\sigma} \rightarrow \vec{\sigma}'}
    = \exp \Bigr [-2 \beta \sigma_i (J \sum_{j \in adj(i)} \sigma_j + B_i) \Bigr ]
\]
Our policy is thus: if the transition lowers the energy, accept it. 
Otherwise, accept it with probability $P_{\vec{\sigma} \rightarrow \vec{\sigma}'}$. 

\newpage
\bibliographystyle{plain}
\bibliography{references}

%%%%%%%%%%%%%%%%%%%%%%%%%%%%%%%%%%%%%%%%%%%%%%%%%%%%%%%%%%%%

\end{document}